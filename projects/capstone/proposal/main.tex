%%%%%%%%%%%%%%%%%%%%%%%%%%%%%%%%%%%%%%%%%
% Wenneker Article
% LaTeX Template
% Version 2.0 (28/2/17)
%
% This template was downloaded from:
% http://www.LaTeXTemplates.com
%
% Authors:
% Vel (vel@LaTeXTemplates.com)
% Frits Wenneker
%
% License:
% CC BY-NC-SA 3.0 (http://creativecommons.org/licenses/by-nc-sa/3.0/)
%
%%%%%%%%%%%%%%%%%%%%%%%%%%%%%%%%%%%%%%%%%

%----------------------------------------------------------------------------------------
%	PACKAGES AND OTHER DOCUMENT CONFIGURATIONS
%----------------------------------------------------------------------------------------

\documentclass[10pt, a4paper]{article}
\usepackage{hyperref}
%%%%%%%%%%%%%%%%%%%%%%%%%%%%%%%%%%%%%%%%%
% Wenneker Article
% Structure Specification File
% Version 1.0 (28/2/17)
%
% This file originates from:
% http://www.LaTeXTemplates.com
%
% Authors:
% Frits Wenneker
% Vel (vel@LaTeXTemplates.com)
%
% License:
% CC BY-NC-SA 3.0 (http://creativecommons.org/licenses/by-nc-sa/3.0/)
%
%%%%%%%%%%%%%%%%%%%%%%%%%%%%%%%%%%%%%%%%%

%----------------------------------------------------------------------------------------
%	PACKAGES AND OTHER DOCUMENT CONFIGURATIONS
%----------------------------------------------------------------------------------------

\usepackage[english]{babel} % English language hyphenation

\usepackage{microtype} % Better typography

\usepackage{amsmath,amsfonts,amsthm} % Math packages for equations

\usepackage[svgnames]{xcolor} % Enabling colors by their 'svgnames'

\usepackage[hang, small, labelfont=bf, up, textfont=it]{caption} % Custom captions under/above tables and figures

\usepackage{booktabs} % Horizontal rules in tables

\usepackage{lastpage} % Used to determine the number of pages in the document (for "Page X of Total")

\usepackage{graphicx} % Required for adding images

\usepackage{enumitem} % Required for customising lists
\setlist{noitemsep} % Remove spacing between bullet/numbered list elements

\usepackage{sectsty} % Enables custom section titles
\allsectionsfont{\usefont{OT1}{phv}{b}{n}} % Change the font of all section commands (Helvetica)

%----------------------------------------------------------------------------------------
%	MARGINS AND SPACING
%----------------------------------------------------------------------------------------

\usepackage{geometry} % Required for adjusting page dimensions

\geometry{
	top=1cm, % Top margin
	bottom=1.5cm, % Bottom margin
	left=2cm, % Left margin
	right=2cm, % Right margin
	includehead, % Include space for a header
	includefoot, % Include space for a footer
	%showframe, % Uncomment to show how the type block is set on the page
}

\setlength{\columnsep}{7mm} % Column separation width

\setlength{\parindent}{0cm}
\setlength{\parskip}{\baselineskip}

%----------------------------------------------------------------------------------------
%	FONTS
%----------------------------------------------------------------------------------------

\usepackage[T1]{fontenc} % Output font encoding for international characters
\usepackage[utf8]{inputenc} % Required for inputting international characters

\usepackage{XCharter} % Use the XCharter font

%----------------------------------------------------------------------------------------
%	HEADERS AND FOOTERS
%----------------------------------------------------------------------------------------

\usepackage{fancyhdr} % Needed to define custom headers/footers
\pagestyle{fancy} % Enables the custom headers/footers

\renewcommand{\headrulewidth}{0.0pt} % No header rule
\renewcommand{\footrulewidth}{0.4pt} % Thin footer rule

\renewcommand{\sectionmark}[1]{\markboth{#1}{}} % Removes the section number from the header when \leftmark is used

%\nouppercase\leftmark % Add this to one of the lines below if you want a section title in the header/footer

% Headers
\lhead{} % Left header
\chead{\textit{\thetitle}} % Center header - currently printing the article title
\rhead{} % Right header

% Footers
\lfoot{} % Left footer
\cfoot{} % Center footer
\rfoot{\footnotesize Page \thepage\ of \pageref{LastPage}} % Right footer, "Page 1 of 2"

\fancypagestyle{firstpage}{ % Page style for the first page with the title
	\fancyhf{}
	\renewcommand{\footrulewidth}{0pt} % Suppress footer rule
}

%----------------------------------------------------------------------------------------
%	TITLE SECTION
%----------------------------------------------------------------------------------------

\newcommand{\authorstyle}[1]{{\large\usefont{OT1}{phv}{b}{n}\color{DarkRed}#1}} % Authors style (Helvetica)

\newcommand{\institution}[1]{{\footnotesize\usefont{OT1}{phv}{m}{sl}\color{Black}#1}} % Institutions style (Helvetica)

\usepackage{titling} % Allows custom title configuration

\newcommand{\HorRule}{\color{DarkGoldenrod}\rule{\linewidth}{1pt}} % Defines the gold horizontal rule around the title

\pretitle{
	\vspace{-30pt} % Move the entire title section up
	\HorRule\vspace{10pt} % Horizontal rule before the title
	\fontsize{32}{36}\usefont{OT1}{phv}{b}{n}\selectfont % Helvetica
	\color{DarkRed} % Text colour for the title and author(s)
}

\posttitle{\par\vskip 15pt} % Whitespace under the title

\preauthor{} % Anything that will appear before \author is printed

\postauthor{ % Anything that will appear after \author is printed
	\vspace{10pt} % Space before the rule
	\par\HorRule % Horizontal rule after the title
	\vspace{20pt} % Space after the title section
}

%----------------------------------------------------------------------------------------
%	ABSTRACT
%----------------------------------------------------------------------------------------

\usepackage{lettrine} % Package to accentuate the first letter of the text (lettrine)
\usepackage{fix-cm}	% Fixes the height of the lettrine

\newcommand{\initial}[1]{ % Defines the command and style for the lettrine
	\lettrine[lines=3,findent=4pt,nindent=0pt]{% Lettrine takes up 3 lines, the text to the right of it is indented 4pt and further indenting of lines 2+ is stopped
		\color{DarkGoldenrod}% Lettrine colour
		{#1}% The letter
	}{}%
}

\usepackage{xstring} % Required for string manipulation

\newcommand{\lettrineabstract}[1]{
	\StrLeft{#1}{1}[\firstletter] % Capture the first letter of the abstract for the lettrine
	\initial{\firstletter}\textbf{\StrGobbleLeft{#1}{1}} % Print the abstract with the first letter as a lettrine and the rest in bold
}

%----------------------------------------------------------------------------------------
%	BIBLIOGRAPHY
%----------------------------------------------------------------------------------------

\usepackage[backend=bibtex,style=authoryear,natbib=true]{biblatex} % Use the bibtex backend with the authoryear citation style (which resembles APA)

\addbibresource{example.bib} % The filename of the bibliography

\usepackage[autostyle=true]{csquotes} % Required to generate language-dependent quotes in the bibliography


%----------------------------------------------------------------------------------------
%	ARTICLE INFORMATION
%----------------------------------------------------------------------------------------

\title{MLND - Project Proposal} % The article title
%\subtitle{Predict Wikipedia Page Traffic}

\author{
	\authorstyle{Louis Tian} % Authors
}

\date{\today} % Add a date here if you would like one to appear underneath the title block, use \today for the current date, leave empty for no date

\begin{document}

\maketitle % Print the title

\thispagestyle{firstpage} % Apply the page style for the first page (no headers and footers)

%----------------------------------------------------------------------------------------
%	ABSTRACT
%----------------------------------------------------------------------------------------


%----------------------------------------------------------------------------------------
%	ARTICLE CONTENTS
%----------------------------------------------------------------------------------------

\section{Domain Background}
Forecasting or time series prediction is an import subfield of machine learning. Since the dawn of humanity, we have been seeking the future telling crystal ball. There are countless applications ranging from traditional science, finance to robotic. While being a very important field, Time series prediction has been neglected in the recent years of up rising of machine learning and AI.

Compared to other fields of machine learning, data in time series problem are relatively scarce. One of the most successful areas in machine learning is computer vision, the prevalence of smart phone and digital camera created a huge wealth of data for computer vision, easily available to researchers. On the other hand, time series data are much harder to come by. Good quality financial market data are not only limited and prohibitively expensive to acquire.

One might argue that the time series prediction problem is an intrinsically harder problem. While in computer vision and speech recognition, the algorithms are merely catching up to human's ability, time series prediction problem has always been on the frontier of human level ability. While a three-year-old could tell the difference between a dog and a cat, only the very best of us are able to make an accurate prediction of future.

This combination of intrinsic difficulty and usefulness makes time series problem a very interesting topic to study

\section{Problem Statement}
The proposed problem is to predict the page visit traffic for a large quantity of Wikipedia web pages based on their historical visits data.This is based on the Kaggle Web Traffic Time Series Forecasting Competition\footnote{https://www.kaggle.com/c/web-traffic-time-series-forecasting}.



\section{Datasets and Inputs}
The data set for this project is available at Kaggle.com. The data set includes daily web traffic data for a total of 145,063 pages for the period from 01/07/2015 to 31/12/2016. For the purpose of this project, only a subset of 100 pages of the original data set will be used.

This data set does not distinguish between traffic values of zero and missing values. A missing value may mean the traffic was zero or that the data is not available for that day.



\section{Solution Statement}
This project proposed to transform this traditional time series problem into a supervised learning problem and use neural nets to forecast the web traffic for a fixed length out of sample period.
Features of the supervised problem can be created using the historical web traffic. Examples of those could include lagged observation for a fixed period of windows, mean/median web traffic for lagged window period.



\section{Benchmark Model}

The traditional approach to time series problem is to fit the time series with ARIMA model. ARIMA model is considered as the most general form time series model in traditional statistically methods. However, the general form of ARIMA requires a significant amount of hyper-parameter tuning makes is unsuitable for our purpose as the benchmark model.
For benchmarking, we will restrict to ARIMA(1,1,0), which basically assumes the time series is generated by a random walk process. Under this setting, the best prediction is simply the latest observed value in the time series.

\section{Evaluation Metrics}

The official evaluation criteria used in the Kaggle competition is SMAPE defined as:

\begin{equation}
SMAPE = \frac{1}{n} \sum_{i} \frac{|{\hat{y}_i- y_i}|}{|\hat{y}_i|+|y_i|}
\end{equation}

\section{Project Design}
The overall work flow of the proposed project is summarised in this section.

\subsubsection*{1. Data Exploration and Visualisation}
Calculate basic statistics about the data set and create some visualisation to understand the fundamental characteristics of the dataset.

\subsubsection*{2. Data Preparation}
Reserve the last 14 days of data as the validation data set

\subsubsection*{3. Calculate the benchmark model performance metric}
ARIMA(1,1,0) benchmark metrics should be calculated.

\subsubsection*{4. Build the models in Tensorflow}
The model will be built in Tensorflow for performance and flexibility.

\subsubsection*{5. Hyperparameter tuning with Grid Search}
Given that only 100 pages will be used,  it is computationally feasible to tune our parameter by using simple grid search.

\subsubsection*{6. Model Evaluation}
Train the model on the entire training set and evaluate the performance by making the 14 days out of sample prediction. Using the predictions and the validation set to calculate the SMAPE score and compare against the benchmark value.

%----------------------------------------------------------------------------------------
%	BIBLIOGRAPHY
%----------------------------------------------------------------------------------------

\printbibliography[title={Bibliography}] % Print the bibliography, section title in curly brackets

%----------------------------------------------------------------------------------------

\end{document}
